\documentclass{article}
\usepackage[margin=1in]{geometry}
\usepackage{hyperref}

\begin{document}

\title{Comprehensive Guide to AI for Alpha Generation}
\author{}
\date{}
\maketitle

\section{Phase 1: Foundations - Essential Finance and AI Concepts}

\subsection{Module 1.1: Core Financial Concepts and Intuition}
This module introduces fundamental financial concepts crucial for AI-driven alpha generation, including Alpha, Beta, Efficient Market Hypothesis (EMH), and the Capital Asset Pricing Model (CAPM). It emphasizes conceptual understanding and practical calculations of Beta and expected returns.

\subsection{Module 1.2: Setting Up Your Local Python Environment}
Focuses on the creation and configuration of an optimized Python environment suited for finance and AI projects, covering essential libraries like NumPy, Pandas, Matplotlib, scikit-learn, yfinance, and statsmodels. Includes instructions for creating structured directories and managing virtual environments.

\subsection{Module 1.3: Accessing and Understanding Market Data (Numerai API)}
Explores accessing financial datasets through APIs, specifically Numerai. This module addresses downloading datasets, understanding data structure, exploring and preprocessing datasets, and understanding the use-cases for Numerai's anonymized financial market data.

\subsection{Module 1.4: Simple Backtesting with Zipline or Backtrader}
Introduces backtesting frameworks Zipline and Backtrader, covering installation, basic concepts of backtesting, strategy implementation, and results interpretation. Special emphasis on analyzing strategy effectiveness and adjustments for improved performance.

\section{Phase 2: Supervised Learning and Factor Models}

\subsection{Module 2.1: Feature Engineering for Finance}
Covers the critical aspects of creating informative financial features from raw data. Techniques such as creating lagged returns, volatility measures, momentum indicators, and economic indicators are explored.

\subsection{Module 2.2: Linear Regression and Factor Models}
Introduces factor models like the Fama-French three-factor model, covering theoretical foundations, factor selection, and the implementation of regression models to explain and predict returns.

\subsection{Module 2.3: Regularization Techniques (Lasso, Ridge)}
Explores regularization techniques to prevent model overfitting. Discusses theory behind Lasso and Ridge regression, their implications, and practical implementation guidelines for financial data.

\subsection{Module 2.4: Factor-Based Stock Selection Strategy}
Guides students through developing and implementing a systematic factor-based stock selection strategy. Covers factor selection, backtesting procedures, and portfolio performance evaluation.

\section{Phase 3: Deep Learning for Time-Series Forecasting}

\subsection{Module 3.1: Understanding Time Series Data}
Introduces fundamental concepts such as stationarity, trends, seasonality, and techniques to preprocess and analyze financial time series data effectively.

\subsection{Module 3.2: LSTM Networks for Stock Price Prediction}
Detailed exploration of Long Short-Term Memory (LSTM) neural networks for capturing temporal dynamics in stock price data. Includes theory, architecture design, hyperparameter tuning, and model evaluation.

\subsection{Module 3.3: CNNs for Pattern Recognition in Financial Data}
Introduces Convolutional Neural Networks (CNNs) and their applications in detecting and leveraging complex patterns in financial time series. Includes data preparation, architecture design, training strategies, and practical considerations.

\subsection{Module 3.4: Practical Deep Learning: Model Training and Optimization}
Discusses CPU-optimized methods for deep learning model training, parameter optimization, and techniques to ensure efficient model performance and reliability on local computational resources.

\section{Phase 4: Reinforcement Learning and Algorithmic Trading}

\subsection{Module 4.1: Introduction to Reinforcement Learning}
Provides a solid theoretical grounding in reinforcement learning, including Markov Decision Processes (MDP), Bellman equations, value functions, and policy optimization.

\subsection{Module 4.2: Implementing Simple Trading Policies (Q-learning)}
Demonstrates practical implementation of Q-learning for creating simple algorithmic trading policies, including strategy design, state/action definition, reward structures, and policy evaluation.

\subsection{Module 4.3: Policy Gradient Methods and Advanced RL Strategies}
Introduces advanced reinforcement learning algorithms like policy gradients, Actor-Critic methods, and Deep Reinforcement Learning for trading. Emphasizes theoretical foundations, implementation strategies, and potential pitfalls.

\subsection{Module 4.4: End-to-End Trading Bot Prototype}
Guides the development of a fully autonomous trading bot using combined RL strategies. Includes the integration of data pipelines, real-time execution, backtesting, and evaluation metrics to finalize a deployable prototype.

\end{document}


